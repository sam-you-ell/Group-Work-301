\documentclass[12pt]{article}
\usepackage{amsmath, amssymb}
\usepackage{graphicx}
\newcommand{\PRLsep}{\noindent\makebox[\linewidth]{\resizebox{0.3333\linewidth}{1pt}{$\bullet$}}\bigskip}

\setlength{\textheight}{9.5in}
\setlength{\textwidth}{7in}
\setlength{\evensidemargin}{-0.2in}
\setlength{\oddsidemargin}{-0.2in}
\setlength{\topmargin}{-0.8in}

\date{}

\begin{document}
\vspace{-0.95cm}

\begin{center}
    {\Large{QHE Progress Report}}\\
    19th March
\end{center}


\subsection*{Aysha Tarafdar - Presenter}
\subsubsection*{Introduction Overview}
Classical Hall Effect: The classical hall effect is a consequence of charged carriers moving in a magnetic field,
which results in a transverse force on the moving charged carriers. The charged carriers move towards one side of
conductor and creates a voltage across it. This voltage is call Hall voltage, after Edwin Hall who discovered this effect in 1879.  
Leads into what the Drude model predicts. I will also give a brief introduction to some of the first experimental findings, as well as 
the motivation behind studying the quantum hall effect.



\vspace{0.5cm}
\PRLsep

\subsection*{Darius Michienzi - Deputy Editor}
The popular article has a plan and an outline as to what is to be covered. The plan is designed such that each section can be covered by different people in the
group that are focusing on specific areas. It will cover briefly cover the classical hall mechanism, so the reader has the necessary understanding for why the
quantum hall effect is different. The link between quantisation and topological invariants will be discussed but kept to a level such that the intended audience
will be able to understand. The final sections will cover the practical applications the quantum hall effect has as well as current areas of research such as its
observation at room temperature in graphene.
\newline

Personally, I will be covering the applications and future study portions of the article and report. This will be how the precision of the quantisation allows the
von Klitzing constant derived from the hall resistivity to be used as a standard for resistance when calibrating devices. Also, how the high mobility in graphene
means even at room temperature the quantum hall effect can be observed.

\vspace{0.5cm}
\PRLsep

\newpage
\subsection*{Samuel Hopkins - Editor}
\subsubsection*{Berry Phase Overview}
We can describe complex systems using Hamiltonians. For a general Hamiltonian, $H(x^a; \lambda^i)$, where $x^a$ are the degrees of freedom of the system, and $\lambda^i$ are fixed parameters, 
that only depend on external variables or apparatus. We are interested in these parameters, and the space they create for our system. 

We can vary $\lambda$ so that we trace out a closed path in parameter space. The adiabatic theorem tells us that if we started in the ground state and varied $\lambda$ sufficiently slowly so that we 
return to the ground state, we simply pick up a phase. This mathematically tells us; 
\begin{equation*}
    |\psi \rangle \to e^{i \gamma}|\psi \rangle
\end{equation*}
This is a phase difference between the same two states that will lead to physical consequences, so we simply cannot ignore it like we would with global phases.
This phase difference has two contributions; 
\begin{enumerate}
    \item Contribution from the Dynamical phase
    \item Contribution from Berry Phase
\end{enumerate} 

\subsubsection*{Plan}
Following from the introduction of Berry Phase, I will detail the computation of the Berry Phase, the Berry Connection and curvature which 
are fundamental concepts when considering the topology of the underlying system in the Integer Quantum Hall Effect, specifically when we consider the form of the 
Hall conductivity and how this leads to our Chern numbers.

\subsubsection*{Current Stage}
Currently I am reading both the David Tong Lectures on QHE, and some of the original papers on Berry Phase. I hope to start writing by the end of week 21.

\vspace{0.5cm}
\PRLsep


\newpage
\subsection*{Toby Rawlings}

\subsubsection*{Landau Levels Overview}

Landau levels are quantised energy levels of a particle in a magnetic field. The energy levels can be labelled by an integer $n$ and are of the form;
\begin{equation*}
    E_n = \hbar \omega_{b}\left(n + \frac{1}{2}\right)
\end{equation*}
Where $\omega_{b}$ is the cyclotron frequency. The energy levels are degenerate as they are not dependent on the wavevector $k$. The introduction of an electric
field and associated potential changes the energy levels to the form
\begin{equation*}
    E_n = \hbar \omega_{b}\left(n + \frac{1}{2}\right) - eE\left(k l_b^2 + \frac{eE}{m\omega_{b}^2}\right) + \frac{m}{2}\frac{E^2}{B^2}
\end{equation*}
In this case, the degeneracy is lifted due to the introduction of a linear dependence on the wavevector.

\noindent Landau Levels are of central importance when discussing QHE as the Hall conductivity takes values that are characterised by integer $n$, when $n$ Landau levels are filled.

\subsubsection*{Plan}
We have two sections;
General introduction to the concept of Landau Levels:
\begin{enumerate}
    \item[$\to$] This section will begin with a classical description of a particle in a magnetic field and we shall discuss key features of the system (cyclotron orbits, etc) and then
        move on to a quantum mechanical description. This will start with the Hamiltonian of an electron in a magnetic field and then give a brief overview of the method of solving for the energy levels.
        The different choices of gauge shall also be discussed. A description of the degeneracy of Landau levels will then be given. The effect of turning on an electric field will be outlined, particularly the lifting of degeneracy.
\end{enumerate}
Relation to integer Quantum Hall effect:
\begin{enumerate}
    \item[$\to$] The significance of the Landau levels to the IQHE will be outlined. Will discuss how filled Landau levels correspond the
        plateaux in conductivity. Will link to the conductivity of a filled Landau levels using the number of states in each Landau level.
\end{enumerate}
\subsubsection*{Current Stage}
Currently at the reading and planning stage. Finding and reading potential references for the report

\vspace{0.5cm}
\PRLsep
\newpage

\subsection*{Tom Berlad}
\subsubsection*{Experimental Overview}

I will firstly write about the experiments that lead to the discovery of the two main types of Quantum Hall Effect, the Integer Quantum Hall Effect,
 and the Fractional Quantum Hall Effect. The Integer QHE was first discovered by Klaus von Klitzing in 1980, who was awarded the Nobel Prize in Physics 
 in 1985 for his discovery. The Fractional QHE was discovered in 1982 by Horst Störmer and Daniel Tsui, which lead to them being awarded the Nobel Prize 
 in Physics 16 years later in 1998, along with physicist Robert Laughlin as he helped develop the theory to explain the complex phenomenon of 
 Fractional QHE. In recent years, Fractional QHE was also observed using Graphene. I will also talk about the MOSFET (metal-oxide-semiconductor 
 field-effect transistor), and how its experimental application led to the discovery of the QHE. The MOSFET includes a metal gate and an oxide 
 insulator, and a p-doped or n-doped silicon semiconductor. An inversion layer is formed at the semiconductor-insulator when a voltage is applied
between source (gates) and the drain (terminals) of the MOSFET, where a thin layer of electrons at a high concentration (negative charge carriers)
forms at the interface between the insulator and semiconductor, and the size of voltage at the gate of the MOSFET controls the number of electrons
in the layer. This causes the formation of a two-dimensional electron gas where electrons are restricted to only be able to freely travel in two 
dimensions, which allowed quantum effects to be observed when the MOSFET is reduced to a very low temperature using liquid helium, as well as applying 
a strong magnetic field. The classical Hall effect was first discovered in 1879 by physicist Edwin Hall, where he demonstrated how electrons concentrate
at the ends of a thin gold plate when a magnetic field is applied perpendicularly, leading to a voltage. Klaus von Klitzing would discover the Integer
QHE using extremely thin layered MOSFET samples developed by Gerhard Dorda and Michael Pepper. This voltage was discovered to be completely quantized with
integer quantum numbers, so as the applied magnetic field is changed, the voltage value only changes into discrete values. Fractional QHE was discovered
using selectively doped low density Gallium Arsenide structures under a perpendicular magnetic field at temperatures as low as 2K using liquid Helium, 
whey they discovered fractional quantum numbered states. This was a surprising discovery as there shouldn't have been states below quantum number v = 1,
and took many years to explain this strange phenomenon. 

\subsubsection*{Current Stage}

I have yet to begin writing my section on the report, however I've been reading and researching how the Quantum Hall Effect was discovered. 
I aim to start writing up the first draft of my section of the report next week. I have currently read the “Quantum Hall Effect TIFR Infosys 
Lectures” written by Professor David Tong, up to Chapter 3; The Fractional Quantum Hall Effect. As I read up to this chapter, I have covered 
the aspects of both Integer and Fractional QHE, which also gives me insight into the processes that lead to the discovery of QHE in general.
I also started reading “25 Years of Quantum Hall Effect (QHE)1
A Personal View on the Discovery, Physics and Applications of this Quantum Effect” written by physicist Klaus von Klitzing who discovered the
Integer QHE. This also gives an insight into the experimentation leading to its discovery from the insight of the actual physicist who worked
to discover it. I also read part of the original 1980 paper by von Klitzing “New Method for High-Accuracy Determination of the Fine-Structure
Constant Based on Quantized Hall Resistance”, giving an insight into the original processes in his research that lead to his discovery of Integer QHE. 


\subsection*{Raul Gonzalez Ramos}
\subsubsection*{Topology Overview}

Quantum Hall effect appears in bidimensional electronic system under the application of strong magnetic fields. 
There are two types, the integer quantum hall effect (IQHE) and the fractional version (FQHE). It seems that the
electron systems which show this effect, specifically the FQHE, contained complex internal structure, known as quantum fluids.
This structure represents a new matter state, and new concepts and ideas were developed to understand these states, 
in particular a branch of mathematics called topology. 

\noindent The intuitive idea is that topology relates systems which share some properties which are invariant under some allowed 
transformations. For example, a doughnut and a mug are equivalent since we can transform one into the other one by isometric 
transformations. 
Topology helps to understand the idea that the plateaus that we saw in the QHE are not altered by the impurities of the systems. 
\subsubsection*{Plan}
\begin{enumerate}
    \item[$\to$] What is topology?: Introduction about mathematical and physical meaning 
    \item[$\to$] Need of using topological order to describe degenerate states: How the idea of topology was necessary to explain and understand the effect. 
    \item[$\to$] Link QHE and topology: Consequences and achievements of using topology, new applications with quantum computing topology.  
\end{enumerate}
\subsubsection*{Current Stage}

General reading about QHE, topology and how topology solves the problem. Understanding the big picture better to deep into more complex relations.  

\vspace{0.5cm}
\PRLsep

\end{document}