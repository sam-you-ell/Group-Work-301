\documentclass[notitlepage, a4paper, prl, aps]{revtex4}

\setlength{\textheight}{10.2in}
\setlength{\textwidth}{7in}
\setlength{\evensidemargin}{-0.2in}
\setlength{\oddsidemargin}{-0.2in}
\setlength{\topmargin}{-0.7in}

\begin{document}

\title{QHE Progress Report - 19th March}

\author{}

\maketitle

\subsection*{Aysha Tarafdar - Presenter}
Aysha will be handling the presentation and the introduction section of the final report. 

\underline{Introduction Section}
\begin{enumerate}
    \item Classical Hall Effect
    \item How it was found experimentally
    \item How it affected the study of solid state physics
    \item Motivation
\end{enumerate}

\subsection*{Darius Michienzi - Deputy Editor}
The popular article has a plan and an outline as to what is to be covered. The plan is designed such that each section can be covered by different people in the 
group that are focusing on specific areas. It will cover briefly cover the classical hall mechanism, so the reader has the necessary understanding for why the 
quantum hall effect is different. The link between quantisation and topological invariants will be discussed but kept to a level such that the intended audience 
will be able to understand. The final sections will cover the practical applications the quantum hall effect has as well as current areas of research such as its 
observation at room temperature in graphene.  
\newline
Personally, I will be covering the applications and future study portions of the article and report. This will be how the precision of the quantisation allows the
 von Klitzing constant derived from the hall resistivity to be used as a standard for resistance when calibrating devices. Also, how the high mobility in graphene 
 means even at room temperature the quantum hall effect can be observed.   

\subsection*{Samuel Hopkins - Editor}

\underline{Final Report Sections by Sam}

\begin{enumerate}
    \item Abstract - to be written with groups advice/direction
    \item Berry Phase
    \item Conclusion - to be written with groups advice/direction
\end{enumerate}

I will be writing the section on the theory of berry phase in the context of QHE, which naturally leads to the Berry connection, curvature.
The main text I am currently using is the lecture notes from David Tong as an introduction into the subject with context \cite{Tong}.

\subsection*{Toby Rawlings}

\subsection*{Tom Berlad}

\subsection*{Raul Gonzalez Ramos}


\bibliography{qhe_prog_rep}


\end{document}