\documentclass[12pt]{article}
\usepackage{amsmath, amssymb}
\usepackage{graphicx}
\newcommand{\PRLsep}{\noindent\makebox[\linewidth]{\resizebox{0.3333\linewidth}{1pt}{$\bullet$}}\bigskip}

\setlength{\textheight}{9.5in}
\setlength{\textwidth}{7in}
\setlength{\evensidemargin}{-0.2in}
\setlength{\oddsidemargin}{-0.2in}
\setlength{\topmargin}{-0.8in}

\date{}

\begin{document}
\vspace{-0.95cm}

\begin{center}
    {\Large{QHE Progress Report}}\\
    19th March
\end{center}


\subsection*{Aysha Tarafdar - Presenter}
Aysha will be handling the presentation and the introduction section of the final report.

\underline{Introduction Section}
\begin{enumerate}
    \item Classical Hall Effect
    \item How it was found experimentally
    \item How it affected the study of solid state physics
    \item Motivation
\end{enumerate}

\vspace{0.5cm}
\PRLsep

\subsection*{Darius Michienzi - Deputy Editor}
The popular article has a plan and an outline as to what is to be covered. The plan is designed such that each section can be covered by different people in the
group that are focusing on specific areas. It will cover briefly cover the classical hall mechanism, so the reader has the necessary understanding for why the
quantum hall effect is different. The link between quantisation and topological invariants will be discussed but kept to a level such that the intended audience
will be able to understand. The final sections will cover the practical applications the quantum hall effect has as well as current areas of research such as its
observation at room temperature in graphene.
\newline

Personally, I will be covering the applications and future study portions of the article and report. This will be how the precision of the quantisation allows the
von Klitzing constant derived from the hall resistivity to be used as a standard for resistance when calibrating devices. Also, how the high mobility in graphene
means even at room temperature the quantum hall effect can be observed.

\vspace{0.5cm}
\PRLsep

\subsection*{Samuel Hopkins - Editor}


\newpage
\subsection*{Toby Rawlings}

\subsubsection*{Landau Levels Overview}

Landau levels are quantised energy levels of a particle in a magnetic field. The energy levels can be labelled by an integer $n$ and are of the form;
\begin{equation*}
    E_n = \hbar \omega_{b}\left(n + \frac{1}{2}\right)
\end{equation*}
Where $\omega_{b}$ is the cyclotron frequency. The energy levels are degenerate as they are not dependent on the wavevector $k$. The introduction of an electric
field and associated potential changes the energy levels to the form
\begin{equation*}
    E_n = \hbar \omega_{b}\left(n + \frac{1}{2}\right) - eE\left(k l_b^2 + \frac{eE}{m\omega_{b}^2}\right) + \frac{m}{2}\frac{E^2}{B^2}
\end{equation*}
In this case, the degeneracy is lifted due to the introduction of a linear dependence on the wavevector.

\noindent Landau Levels are of central importance when discussing QHE as the Hall conductivity takes values that are characterised by integer $n$, when $n$ Landau levels are filled.

\subsubsection*{Plan}
We have two sections;
General introduction to the concept of Landau Levels:
\begin{enumerate}
    \item[$\to$] This section will begin with a classical description of a particle in a magnetic field and we shall discuss key features of the system (cyclotron orbits, etc) and then
        move on to a quantum mechanical description. This will start with the Hamiltonian of an electron in a magnetic field and then give a brief overview of the method of solving for the energy levels.
        The different choices of gauge shall also be discussed. A description of the degeneracy of Landau levels will then be given. The effect of turning on an electric field will be outlined, particularly the lifting of degeneracy.
\end{enumerate}
Relation to integer Quantum Hall effect:
\begin{enumerate}
    \item[$\to$] The significance of the Landau levels to the IQHE will be outlined. Will discuss how filled Landau levels correspond the
        plateaux in conductivity. Will link to the conductivity of a filled Landau levels using the number of states in each Landau level.
\end{enumerate}
\subsubsection*{Current Stage}
Currently at the reading and planning stage. Finding and reading potential references for the report

\vspace{0.5cm}
\PRLsep
\newpage

\subsection*{Tom Berlad}

\subsection*{Raul Gonzalez Ramos}



\end{document}